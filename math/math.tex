\documentclass{article}
\usepackage[utf8]{inputenc}
\usepackage[russian]{babel}
\usepackage{hyperref}
\usepackage{amsmath}

\setlength{\parindent}{0pt}

\begin{document}

\section{Линейная Алгебра}

\subsection{Линейное (векторное) пространство}

Линейное пространство -- это набор элементов (векторов), для которых определена операция сложения и умножения на число. Эти операции должны подчиняться набору аксиом. \\

Детальная статья в Википедии: \href{https://ru.wikipedia.org/wiki/%D0%92%D0%B5%D0%BA%D1%82%D0%BE%D1%80%D0%BD%D0%BE%D0%B5_%D0%BF%D1%80%D0%BE%D1%81%D1%82%D1%80%D0%B0%D0%BD%D1%81%D1%82%D0%B2%D0%BE}{Векторное пространство}. В этой же статье:
\begin{itemize}
	\item Линейная комбинация векторов
	\item Подпространство
	\item Линейная (не)зависимость векторов
	\item Базис, размерность (ранг)
	\item \href{https://ru.wikipedia.org/wiki/%D0%9D%D0%BE%D1%80%D0%BC%D0%B0_(%D0%BC%D0%B0%D1%82%D0%B5%D0%BC%D0%B0%D1%82%D0%B8%D0%BA%D0%B0)}{Норма вектора}
\end{itemize}

\subsection{Системы линейных уравнений}

Урок на Stepik: \href{https://stepik.org/course/2461/syllabus}{Существование систем линейных уравнений}. \\

Рассмотрим следующую систему линейных уравнений:

\[ \begin{array}{c}
	a_{11} x_{1} + a_{12} x_2 + a_{13} x_3 = b_1 \\
	a_{21} x_{1} + a_{22} x_2 + a_{23} x_3 = b_2 \\
	a_{31} x_{1} + a_{32} x_2 + a_{33} x_3 = b_3 \\
\end{array} \]

В такой системе количество уравнений совпадает с количеством неизвестных. Запишем систему в следующем виде:

\[
	x_1 \cdot \begin{pmatrix} a_{11} \\ a_{21} \\ a_{31} \end{pmatrix} + 
	x_2 \cdot \begin{pmatrix} a_{12} \\ a_{22} \\ a_{32} \end{pmatrix} + 
	x_3 \cdot \begin{pmatrix} a_{13} \\ a_{23} \\ a_{33} \end{pmatrix} =
	\begin{pmatrix} b_1 \\ b_2 \\ b_3 \end{pmatrix} 
\]

В таком виде задачу о нахождении решения данной системы можно рассматривать как задачу о представлении вектора $\mathbf{b}$ в виде линейной комбинации векторов $\mathbf{a_1}$, $\mathbf{a_2}$ и $\mathbf{a_3}$. \\

Если вектора $\mathbf{a_1}$, $\mathbf{a_2}$ и $\mathbf{a_3}$ образуют базис, то решение у такой системы существует при любом векторе $\mathbf{b}$, причем такое решение будет единственным. Если же эти вектора базис не образуют, то решение у системы будет существовать только в том случае, если вектор $\mathbf{b}$ будет принадлежать подпространству, пораждаемому векторами $\mathbf{a_1}$, $\mathbf{a_2}$ и $\mathbf{a_3}$, причем решений в таком случае будет бесконечно много. \\

Аналогичные утверждения верны и для системы линейных уравнений с $n$ уравнениями и $n$ неизвестными.

\end{document}
