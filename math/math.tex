\documentclass{article}
\usepackage[utf8]{inputenc}
\usepackage[russian]{babel}
\usepackage{hyperref}
\usepackage{amsmath}
\usepackage{amsfonts}

\setlength{\parindent}{0pt}

\begin{document}

\tableofcontents

\section{Линейная Алгебра}

\subsection{Линейное (векторное) пространство}

Линейное пространство -- это набор элементов (векторов), для которых определена операция сложения и умножения на число. Эти операции должны подчиняться определенному набору аксиом. \\

Детальная статья в Википедии (в которой в том числе перечислены все аксиомы): \href{https://ru.wikipedia.org/wiki/%D0%92%D0%B5%D0%BA%D1%82%D0%BE%D1%80%D0%BD%D0%BE%D0%B5_%D0%BF%D1%80%D0%BE%D1%81%D1%82%D1%80%D0%B0%D0%BD%D1%81%D1%82%D0%B2%D0%BE}{Векторное пространство}. \\

Примеры линейных (векторных) пространств:

\begin{itemize}
	\item Множество векторов на плоскости.
	\item Множество всех матриц размерности $m \times n$.
	\item Множество всех многочленов степени не выше $n$: \\ $f(x) = a_0 + a_1 \cdot x + a_2 \cdot x^2 + \dots + a_n \cdot x^n$
\end{itemize}

\subsection{Линейная зависимость и независимость векторов}

Рассмотрим набор векторов $v_1, v_2, \dots, v_n$. Данный набор векторов является \textbf{линейно зависимым}, если существуют такие числа $a_1, a_2, \dots, a_n$, что хотя бы одно из этих чисел не равно нулю, и при этом выполнено равенство

$$ a_1 \cdot v_1 + a_2 \cdot v_2 + \dots + a_n \cdot v_n = 0 $$

Если же равенство выше равно нулю только в том случае, если все числа $a_1, a_2, \dots, a_n$ равны нулю, то данный набор векторов называется \textbf{линейно независимым}.

\subsection{Размерность линейного пространства. Базис.}

Рассмотрим линейное пространство $L$. Рассмотрим набор из $n$ векторов $$v_1, v_2, \dots, v_n$$ принадлежащих этому пространству. Предположим, что этот набор векторов является линейно независимым и при этом любой набор из $n + 1$ векторов из этого же пространства является линейно зависимым. В таком случае $L$ называется $n$-мерным векторным пространством, и размерность этого пространства $dim(L) = n$. \\

Вектора $v_1, v_2, \dots, v_n$ образуют \textbf{базис} этого линейного пространства. Любой вектор $u \in L$ можно единственным образом представить в виде линейной комбинации векторов базиса. \\

\subsection{Подпространство}

Множество векторов $$u_1, u_2, \cdots, u_m$$ принадлежащих $L$, образует \textbf{подпространство} $M$, если для этих векторов заданы те же операции сложения и умножения на число, что и в исходном пространстве, и при этом любой вектор $u$, который является результатом выполнения этих операций, также принадлежит $M$.

\subsection{Системы линейных уравнений}

Урок на Stepik: \href{https://stepik.org/course/2461/syllabus}{Существование систем линейных уравнений}. \\

\subsubsection{Частный случай. Число уравнений равно числу неизвестных.}

Рассмотрим следующую систему линейных уравнений:

\[ \begin{array}{c}
	a_{11} x_{1} + a_{12} x_2 + a_{13} x_3 = b_1 \\
	a_{21} x_{1} + a_{22} x_2 + a_{23} x_3 = b_2 \\
	a_{31} x_{1} + a_{32} x_2 + a_{33} x_3 = b_3 \\
\end{array} \]

В такой системе количество уравнений совпадает с количеством неизвестных. Запишем систему в следующем виде:

\[
	x_1 \cdot \begin{pmatrix} a_{11} \\ a_{21} \\ a_{31} \end{pmatrix} + 
	x_2 \cdot \begin{pmatrix} a_{12} \\ a_{22} \\ a_{32} \end{pmatrix} + 
	x_3 \cdot \begin{pmatrix} a_{13} \\ a_{23} \\ a_{33} \end{pmatrix} =
	\begin{pmatrix} b_1 \\ b_2 \\ b_3 \end{pmatrix} 
\]

В таком виде задачу о нахождении решения данной системы можно рассматривать как задачу о представлении вектора $\mathbf{b}$ в виде линейной комбинации векторов $\mathbf{a_1}$, $\mathbf{a_2}$ и $\mathbf{a_3}$. \\

Если вектора $\mathbf{a_1}$, $\mathbf{a_2}$ и $\mathbf{a_3}$ образуют базис, то решение у такой системы существует при любом векторе $\mathbf{b}$, причем такое решение будет единственным. Если же эти вектора базис не образуют, то решение у системы будет существовать только в том случае, если вектор $\mathbf{b}$ будет принадлежать подпространству, пораждаемому векторами $\mathbf{a_1}$, $\mathbf{a_2}$ и $\mathbf{a_3}$, причем решений в таком случае будет бесконечно много. \\

Аналогичные утверждения верны и для системы линейных уравнений с $n$ уравнениями и $n$ неизвестными. \\

\subsubsection{Общий случай}

Рассмотрим теперь более общий случай. А именно, рассмотрим систему, состояющую из $n$ линейных уравнений с $m$ неизвестными:

\[ \begin{array}{c}
	a_{11} x_{1} + a_{12} x_2 + \cdots + a_{1m} x_m = b_1 \\
	a_{21} x_{1} + a_{22} x_2 + \cdots + a_{2m} x_m = b_2 \\
	\cdots \\
	a_{n1} x_{1} + a_{n2} x_2 + \cdots + a_{nm} x_m = b_n \\
\end{array} \]

Перепишем систему в следующем виде:

\[
	x_1 \cdot \begin{pmatrix} a_{11} \\ a_{21} \\ \cdots \\ a_{n1} \end{pmatrix} + 
	x_2 \cdot \begin{pmatrix} a_{12} \\ a_{22} \\ \cdots \\ a_{n2} \end{pmatrix} +
	\cdots +
	x_m \cdot \begin{pmatrix} a_{1m} \\ a_{2m} \\ \cdots \\ a_{nm} \end{pmatrix} =
	\begin{pmatrix} b_1 \\ b_2 \\ \cdots\\  b_n \end{pmatrix} 
\]

В таком виде задачу о нахождении решения для данной системы уравнений можно рассматривать как задачу о представлении вектора $\mathbf{b}$ в виде линейной комбинации векторов $\mathbf{a_1}, \mathbf{a_2}, \cdots, \mathbf{a_m}$, каждый из которых является элементом $n$-мерного линейного пространства. \\

Рассмотрим линейное подпространство минимальной размерности, которое содержит все эти $m$ векторов. Такое подпространство также называется линейной оболочкой, образуемой данными векторами. Размерность такого подпространства (линейной оболочки) называется \textbf{рангом} системы линейных уравнений. \\

Касательно существования решения для системы таких уравнений. Возможны два случая: 

\begin{itemize}
	\item Если вектор $b$ не принадлежит данной линейной оболочке, то решений у системы нет.
	\item Если вектор $b$ принадлежит данной линейной оболочке то, решение существует. При этом если $n = m$, то решение будет единственным, так как набор векторов $\mathbf{a_1}, \mathbf{a_2}, \cdots, \mathbf{a_m}$ будет образовывать базис. Если же число векторов больше, чем размерность линейной оболочки, то система будет иметь бесконечно много решений.
\end{itemize}

\subsection{Решение систем линейных уравнений. Метод Гаусса.}

Урок на Stepik: \href{https://stepik.org/lesson/9582/step/1?unit=23533}{Решение систем линейных алгебраических уравнений. Метод Гаусса}. \\

Основная идея метода Гаусса заключается в том, чтобы c помощью операций сложения и умножения на число последовательно исключать переменные, приводя матрицу коэффициентов к треугольному (диагональному виду). Имея матрицу в таком виде, можно затем последовательно найти значения всех неизвестных.

\subsection{Евклидово пространство}

\subsubsection{Скалярное произведение}

Для двух векторов $u, v$, принадлежащих некоторому линейному пространству $L$, скалярным произведением называется операция, которая этим двум векторам сопоставляет некоторое вещественное число: $$(u, v) = c: \ c \in \mathbb{R}.$$

При этом такая операция должна удовлетворять 4-м аксиомам:

\begin{enumerate}
	\item $(x, y) = (y, x)$
	\item $(\lambda x, y) = \lambda \cdot (x, y) \ \forall \lambda \in \mathbb{R}$
	\item $(x_1 + x_2, y) = (x_1, y) + (x_2, y)$
	\item $(x, x) \ge 0$
\end{enumerate}

Линейное (векторное) пространство с введенной на нем вышеописанной операцией скалярного произведения, называется \textbf{Евклидовым пространством}. \\

Для векторов $x = (x_1, x_2, \dots, x_n), y = (y_1, y_2, \dots, y_n) \in \mathbb{R}^n$ примером скалярного произведения может выступать сумма произведений их координат:

$$ (u, v) = x_1 \cdot y_1 + x_2 \cdot y_2 + \dots + x_n \cdot y_n$$

\subsubsection{Угол между векторами, длина вектора}

Для векторов на плоскости скалярное произведение можно ввести следующим образом:

$$(x, y) = |x| \cdot |y| \cdot \cos(x, y)$$

Выразим отсюда косинус угла между векторами:

$$ \cos(x, y) = \frac{(x, y)}{|x| \cdot |y|} = \cos (\alpha) $$

Из этого выражения получим, что \textbf{угол между векторами} можно найти следующим образом:

$$ \alpha = \arccos\left(\frac{(x, y)}{|x| \cdot |y|}\right) $$

Рассмотрим теперь выражение для скалярного произведения, в котором $y = x$.

$$ (x, x) = |x| \cdot |x| \cdot \cos(x, x) = {|x|}^2 $$

Отсюда получим, что \textbf{длина вектора} $x$ есть

$$ |x| = \sqrt{(x, x)} $$

То есть, если мы знаем, чему равно скалярное произведение, то мы можем найти угол между векторами, а также длину вектора. \\

Данные понятия можно обобщить на случай произвольного векторного пространства. А именно, длину произвольного вектора $x$ можно определить как квадратный корень из скалярного произведения этого вектора на самого себя:

$$ |x| = \sqrt{(x, x)} $$

Угол $\phi$ между произвольными векторами $x, y$ есть

$$ \phi = \arccos\left(\frac{(x, y)}{|x| \cdot |y|}\right) $$

Для векторов на плоскости скалярное произведение будет равно нулю, если векторы ортогональны, то есть $\phi = \frac{\pi}{2}$. Два произвольных вектора будем называть \textbf{ортогональными}, если их скалярное произведение равно нулю.

\subsection{Операторы и базис}

Урок на Stepik: \href{https://stepik.org/lesson/9584/step/1?unit=23534}{Ортогональный базис}.

\subsubsection{Ортонормированный базис}

Рассмотрим набор векторов $\{e_1, e_2, \dots e_n\}$ таких, что

\begin{enumerate}
	\item $(e_i, e_j) = 0 \ \forall i, j: \ i \ne j$.
	\item $(e_i, e_j) = 1 \ \forall i, j: \ i = j$.
\end{enumerate}

Такой набор векторов называется ортонормированным набором векторов в линейном пространстве со скалярным произведением. \\

Уроки на Stepik:

\begin{itemize}
	\item \href{https://stepik.org/lesson/9584/step/6?unit=23534}{Как произвольный базис преобразовать в ортонормированный}
	\item \href{https://stepik.org/lesson/9584/step/13?unit=23534}{Метод наименьших квадратов}
\end{itemize}

\subsection{Линейные операторы}

Оператор $A: \ A \cdot x = y$ называется линейным оператором, если выполнены следующие аксиомы:

\begin{enumerate}
	\item $A \cdot (x_1 + x_2) = A \cdot x_1 + A \cdot x_2$
	\item $A (\lambda x) = \lambda \cdot Ax$
\end{enumerate}

\subsubsection{Ядро и образ оператора}

...

\subsubsection{Собственные числа и собственные векторы}

...

\subsection{Определитель матрицы}

Урок на Stepik: \href{https://stepik.org/lesson/44077/step/1?unit=21901}{Определитель матрицы}.

\end{document}














